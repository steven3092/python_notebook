
% Default to the notebook output style

    


% Inherit from the specified cell style.




    
\documentclass[11pt]{article}

    
    
    \usepackage[T1]{fontenc}
    % Nicer default font than Computer Modern for most use cases
    \usepackage{palatino}

    % Basic figure setup, for now with no caption control since it's done
    % automatically by Pandoc (which extracts ![](path) syntax from Markdown).
    \usepackage{graphicx}
    % We will generate all images so they have a width \maxwidth. This means
    % that they will get their normal width if they fit onto the page, but
    % are scaled down if they would overflow the margins.
    \makeatletter
    \def\maxwidth{\ifdim\Gin@nat@width>\linewidth\linewidth
    \else\Gin@nat@width\fi}
    \makeatother
    \let\Oldincludegraphics\includegraphics
    % Set max figure width to be 80% of text width, for now hardcoded.
    \renewcommand{\includegraphics}[1]{\Oldincludegraphics[width=.8\maxwidth]{#1}}
    % Ensure that by default, figures have no caption (until we provide a
    % proper Figure object with a Caption API and a way to capture that
    % in the conversion process - todo).
    \usepackage{caption}
    \DeclareCaptionLabelFormat{nolabel}{}
    \captionsetup{labelformat=nolabel}

    \usepackage{adjustbox} % Used to constrain images to a maximum size 
    \usepackage{xcolor} % Allow colors to be defined
    \usepackage{enumerate} % Needed for markdown enumerations to work
    \usepackage{geometry} % Used to adjust the document margins
    \usepackage{amsmath} % Equations
    \usepackage{amssymb} % Equations
    \usepackage{textcomp} % defines textquotesingle
    % Hack from http://tex.stackexchange.com/a/47451/13684:
    \AtBeginDocument{%
        \def\PYZsq{\textquotesingle}% Upright quotes in Pygmentized code
    }
    \usepackage{upquote} % Upright quotes for verbatim code
    \usepackage{eurosym} % defines \euro
    \usepackage[mathletters]{ucs} % Extended unicode (utf-8) support
    \usepackage[utf8x]{inputenc} % Allow utf-8 characters in the tex document
    \usepackage{fancyvrb} % verbatim replacement that allows latex
    \usepackage{grffile} % extends the file name processing of package graphics 
                         % to support a larger range 
    % The hyperref package gives us a pdf with properly built
    % internal navigation ('pdf bookmarks' for the table of contents,
    % internal cross-reference links, web links for URLs, etc.)
    \usepackage{hyperref}
    \usepackage{longtable} % longtable support required by pandoc >1.10
    \usepackage{booktabs}  % table support for pandoc > 1.12.2
    \usepackage[normalem]{ulem} % ulem is needed to support strikethroughs (\sout)
                                % normalem makes italics be italics, not underlines
    

    
    
    % Colors for the hyperref package
    \definecolor{urlcolor}{rgb}{0,.145,.698}
    \definecolor{linkcolor}{rgb}{.71,0.21,0.01}
    \definecolor{citecolor}{rgb}{.12,.54,.11}

    % ANSI colors
    \definecolor{ansi-black}{HTML}{3E424D}
    \definecolor{ansi-black-intense}{HTML}{282C36}
    \definecolor{ansi-red}{HTML}{E75C58}
    \definecolor{ansi-red-intense}{HTML}{B22B31}
    \definecolor{ansi-green}{HTML}{00A250}
    \definecolor{ansi-green-intense}{HTML}{007427}
    \definecolor{ansi-yellow}{HTML}{DDB62B}
    \definecolor{ansi-yellow-intense}{HTML}{B27D12}
    \definecolor{ansi-blue}{HTML}{208FFB}
    \definecolor{ansi-blue-intense}{HTML}{0065CA}
    \definecolor{ansi-magenta}{HTML}{D160C4}
    \definecolor{ansi-magenta-intense}{HTML}{A03196}
    \definecolor{ansi-cyan}{HTML}{60C6C8}
    \definecolor{ansi-cyan-intense}{HTML}{258F8F}
    \definecolor{ansi-white}{HTML}{C5C1B4}
    \definecolor{ansi-white-intense}{HTML}{A1A6B2}

    % commands and environments needed by pandoc snippets
    % extracted from the output of `pandoc -s`
    \providecommand{\tightlist}{%
      \setlength{\itemsep}{0pt}\setlength{\parskip}{0pt}}
    \DefineVerbatimEnvironment{Highlighting}{Verbatim}{commandchars=\\\{\}}
    % Add ',fontsize=\small' for more characters per line
    \newenvironment{Shaded}{}{}
    \newcommand{\KeywordTok}[1]{\textcolor[rgb]{0.00,0.44,0.13}{\textbf{{#1}}}}
    \newcommand{\DataTypeTok}[1]{\textcolor[rgb]{0.56,0.13,0.00}{{#1}}}
    \newcommand{\DecValTok}[1]{\textcolor[rgb]{0.25,0.63,0.44}{{#1}}}
    \newcommand{\BaseNTok}[1]{\textcolor[rgb]{0.25,0.63,0.44}{{#1}}}
    \newcommand{\FloatTok}[1]{\textcolor[rgb]{0.25,0.63,0.44}{{#1}}}
    \newcommand{\CharTok}[1]{\textcolor[rgb]{0.25,0.44,0.63}{{#1}}}
    \newcommand{\StringTok}[1]{\textcolor[rgb]{0.25,0.44,0.63}{{#1}}}
    \newcommand{\CommentTok}[1]{\textcolor[rgb]{0.38,0.63,0.69}{\textit{{#1}}}}
    \newcommand{\OtherTok}[1]{\textcolor[rgb]{0.00,0.44,0.13}{{#1}}}
    \newcommand{\AlertTok}[1]{\textcolor[rgb]{1.00,0.00,0.00}{\textbf{{#1}}}}
    \newcommand{\FunctionTok}[1]{\textcolor[rgb]{0.02,0.16,0.49}{{#1}}}
    \newcommand{\RegionMarkerTok}[1]{{#1}}
    \newcommand{\ErrorTok}[1]{\textcolor[rgb]{1.00,0.00,0.00}{\textbf{{#1}}}}
    \newcommand{\NormalTok}[1]{{#1}}
    
    % Additional commands for more recent versions of Pandoc
    \newcommand{\ConstantTok}[1]{\textcolor[rgb]{0.53,0.00,0.00}{{#1}}}
    \newcommand{\SpecialCharTok}[1]{\textcolor[rgb]{0.25,0.44,0.63}{{#1}}}
    \newcommand{\VerbatimStringTok}[1]{\textcolor[rgb]{0.25,0.44,0.63}{{#1}}}
    \newcommand{\SpecialStringTok}[1]{\textcolor[rgb]{0.73,0.40,0.53}{{#1}}}
    \newcommand{\ImportTok}[1]{{#1}}
    \newcommand{\DocumentationTok}[1]{\textcolor[rgb]{0.73,0.13,0.13}{\textit{{#1}}}}
    \newcommand{\AnnotationTok}[1]{\textcolor[rgb]{0.38,0.63,0.69}{\textbf{\textit{{#1}}}}}
    \newcommand{\CommentVarTok}[1]{\textcolor[rgb]{0.38,0.63,0.69}{\textbf{\textit{{#1}}}}}
    \newcommand{\VariableTok}[1]{\textcolor[rgb]{0.10,0.09,0.49}{{#1}}}
    \newcommand{\ControlFlowTok}[1]{\textcolor[rgb]{0.00,0.44,0.13}{\textbf{{#1}}}}
    \newcommand{\OperatorTok}[1]{\textcolor[rgb]{0.40,0.40,0.40}{{#1}}}
    \newcommand{\BuiltInTok}[1]{{#1}}
    \newcommand{\ExtensionTok}[1]{{#1}}
    \newcommand{\PreprocessorTok}[1]{\textcolor[rgb]{0.74,0.48,0.00}{{#1}}}
    \newcommand{\AttributeTok}[1]{\textcolor[rgb]{0.49,0.56,0.16}{{#1}}}
    \newcommand{\InformationTok}[1]{\textcolor[rgb]{0.38,0.63,0.69}{\textbf{\textit{{#1}}}}}
    \newcommand{\WarningTok}[1]{\textcolor[rgb]{0.38,0.63,0.69}{\textbf{\textit{{#1}}}}}
    
    
    % Define a nice break command that doesn't care if a line doesn't already
    % exist.
    \def\br{\hspace*{\fill} \\* }
    % Math Jax compatability definitions
    \def\gt{>}
    \def\lt{<}
    % Document parameters
    \title{Structure\_de\_donnees\_integrees}
    
    
    

    % Pygments definitions
    
\makeatletter
\def\PY@reset{\let\PY@it=\relax \let\PY@bf=\relax%
    \let\PY@ul=\relax \let\PY@tc=\relax%
    \let\PY@bc=\relax \let\PY@ff=\relax}
\def\PY@tok#1{\csname PY@tok@#1\endcsname}
\def\PY@toks#1+{\ifx\relax#1\empty\else%
    \PY@tok{#1}\expandafter\PY@toks\fi}
\def\PY@do#1{\PY@bc{\PY@tc{\PY@ul{%
    \PY@it{\PY@bf{\PY@ff{#1}}}}}}}
\def\PY#1#2{\PY@reset\PY@toks#1+\relax+\PY@do{#2}}

\expandafter\def\csname PY@tok@ne\endcsname{\let\PY@bf=\textbf\def\PY@tc##1{\textcolor[rgb]{0.82,0.25,0.23}{##1}}}
\expandafter\def\csname PY@tok@sb\endcsname{\def\PY@tc##1{\textcolor[rgb]{0.73,0.13,0.13}{##1}}}
\expandafter\def\csname PY@tok@mo\endcsname{\def\PY@tc##1{\textcolor[rgb]{0.40,0.40,0.40}{##1}}}
\expandafter\def\csname PY@tok@ni\endcsname{\let\PY@bf=\textbf\def\PY@tc##1{\textcolor[rgb]{0.60,0.60,0.60}{##1}}}
\expandafter\def\csname PY@tok@nl\endcsname{\def\PY@tc##1{\textcolor[rgb]{0.63,0.63,0.00}{##1}}}
\expandafter\def\csname PY@tok@ch\endcsname{\let\PY@it=\textit\def\PY@tc##1{\textcolor[rgb]{0.25,0.50,0.50}{##1}}}
\expandafter\def\csname PY@tok@ss\endcsname{\def\PY@tc##1{\textcolor[rgb]{0.10,0.09,0.49}{##1}}}
\expandafter\def\csname PY@tok@cp\endcsname{\def\PY@tc##1{\textcolor[rgb]{0.74,0.48,0.00}{##1}}}
\expandafter\def\csname PY@tok@gr\endcsname{\def\PY@tc##1{\textcolor[rgb]{1.00,0.00,0.00}{##1}}}
\expandafter\def\csname PY@tok@nv\endcsname{\def\PY@tc##1{\textcolor[rgb]{0.10,0.09,0.49}{##1}}}
\expandafter\def\csname PY@tok@na\endcsname{\def\PY@tc##1{\textcolor[rgb]{0.49,0.56,0.16}{##1}}}
\expandafter\def\csname PY@tok@no\endcsname{\def\PY@tc##1{\textcolor[rgb]{0.53,0.00,0.00}{##1}}}
\expandafter\def\csname PY@tok@nb\endcsname{\def\PY@tc##1{\textcolor[rgb]{0.00,0.50,0.00}{##1}}}
\expandafter\def\csname PY@tok@vc\endcsname{\def\PY@tc##1{\textcolor[rgb]{0.10,0.09,0.49}{##1}}}
\expandafter\def\csname PY@tok@nt\endcsname{\let\PY@bf=\textbf\def\PY@tc##1{\textcolor[rgb]{0.00,0.50,0.00}{##1}}}
\expandafter\def\csname PY@tok@sr\endcsname{\def\PY@tc##1{\textcolor[rgb]{0.73,0.40,0.53}{##1}}}
\expandafter\def\csname PY@tok@go\endcsname{\def\PY@tc##1{\textcolor[rgb]{0.53,0.53,0.53}{##1}}}
\expandafter\def\csname PY@tok@kc\endcsname{\let\PY@bf=\textbf\def\PY@tc##1{\textcolor[rgb]{0.00,0.50,0.00}{##1}}}
\expandafter\def\csname PY@tok@ow\endcsname{\let\PY@bf=\textbf\def\PY@tc##1{\textcolor[rgb]{0.67,0.13,1.00}{##1}}}
\expandafter\def\csname PY@tok@kp\endcsname{\def\PY@tc##1{\textcolor[rgb]{0.00,0.50,0.00}{##1}}}
\expandafter\def\csname PY@tok@gi\endcsname{\def\PY@tc##1{\textcolor[rgb]{0.00,0.63,0.00}{##1}}}
\expandafter\def\csname PY@tok@cs\endcsname{\let\PY@it=\textit\def\PY@tc##1{\textcolor[rgb]{0.25,0.50,0.50}{##1}}}
\expandafter\def\csname PY@tok@ge\endcsname{\let\PY@it=\textit}
\expandafter\def\csname PY@tok@gt\endcsname{\def\PY@tc##1{\textcolor[rgb]{0.00,0.27,0.87}{##1}}}
\expandafter\def\csname PY@tok@cpf\endcsname{\let\PY@it=\textit\def\PY@tc##1{\textcolor[rgb]{0.25,0.50,0.50}{##1}}}
\expandafter\def\csname PY@tok@gu\endcsname{\let\PY@bf=\textbf\def\PY@tc##1{\textcolor[rgb]{0.50,0.00,0.50}{##1}}}
\expandafter\def\csname PY@tok@c1\endcsname{\let\PY@it=\textit\def\PY@tc##1{\textcolor[rgb]{0.25,0.50,0.50}{##1}}}
\expandafter\def\csname PY@tok@vi\endcsname{\def\PY@tc##1{\textcolor[rgb]{0.10,0.09,0.49}{##1}}}
\expandafter\def\csname PY@tok@nn\endcsname{\let\PY@bf=\textbf\def\PY@tc##1{\textcolor[rgb]{0.00,0.00,1.00}{##1}}}
\expandafter\def\csname PY@tok@c\endcsname{\let\PY@it=\textit\def\PY@tc##1{\textcolor[rgb]{0.25,0.50,0.50}{##1}}}
\expandafter\def\csname PY@tok@il\endcsname{\def\PY@tc##1{\textcolor[rgb]{0.40,0.40,0.40}{##1}}}
\expandafter\def\csname PY@tok@s\endcsname{\def\PY@tc##1{\textcolor[rgb]{0.73,0.13,0.13}{##1}}}
\expandafter\def\csname PY@tok@se\endcsname{\let\PY@bf=\textbf\def\PY@tc##1{\textcolor[rgb]{0.73,0.40,0.13}{##1}}}
\expandafter\def\csname PY@tok@mf\endcsname{\def\PY@tc##1{\textcolor[rgb]{0.40,0.40,0.40}{##1}}}
\expandafter\def\csname PY@tok@nc\endcsname{\let\PY@bf=\textbf\def\PY@tc##1{\textcolor[rgb]{0.00,0.00,1.00}{##1}}}
\expandafter\def\csname PY@tok@nf\endcsname{\def\PY@tc##1{\textcolor[rgb]{0.00,0.00,1.00}{##1}}}
\expandafter\def\csname PY@tok@err\endcsname{\def\PY@bc##1{\setlength{\fboxsep}{0pt}\fcolorbox[rgb]{1.00,0.00,0.00}{1,1,1}{\strut ##1}}}
\expandafter\def\csname PY@tok@m\endcsname{\def\PY@tc##1{\textcolor[rgb]{0.40,0.40,0.40}{##1}}}
\expandafter\def\csname PY@tok@s1\endcsname{\def\PY@tc##1{\textcolor[rgb]{0.73,0.13,0.13}{##1}}}
\expandafter\def\csname PY@tok@kr\endcsname{\let\PY@bf=\textbf\def\PY@tc##1{\textcolor[rgb]{0.00,0.50,0.00}{##1}}}
\expandafter\def\csname PY@tok@sc\endcsname{\def\PY@tc##1{\textcolor[rgb]{0.73,0.13,0.13}{##1}}}
\expandafter\def\csname PY@tok@o\endcsname{\def\PY@tc##1{\textcolor[rgb]{0.40,0.40,0.40}{##1}}}
\expandafter\def\csname PY@tok@kn\endcsname{\let\PY@bf=\textbf\def\PY@tc##1{\textcolor[rgb]{0.00,0.50,0.00}{##1}}}
\expandafter\def\csname PY@tok@si\endcsname{\let\PY@bf=\textbf\def\PY@tc##1{\textcolor[rgb]{0.73,0.40,0.53}{##1}}}
\expandafter\def\csname PY@tok@bp\endcsname{\def\PY@tc##1{\textcolor[rgb]{0.00,0.50,0.00}{##1}}}
\expandafter\def\csname PY@tok@gs\endcsname{\let\PY@bf=\textbf}
\expandafter\def\csname PY@tok@sx\endcsname{\def\PY@tc##1{\textcolor[rgb]{0.00,0.50,0.00}{##1}}}
\expandafter\def\csname PY@tok@k\endcsname{\let\PY@bf=\textbf\def\PY@tc##1{\textcolor[rgb]{0.00,0.50,0.00}{##1}}}
\expandafter\def\csname PY@tok@kt\endcsname{\def\PY@tc##1{\textcolor[rgb]{0.69,0.00,0.25}{##1}}}
\expandafter\def\csname PY@tok@cm\endcsname{\let\PY@it=\textit\def\PY@tc##1{\textcolor[rgb]{0.25,0.50,0.50}{##1}}}
\expandafter\def\csname PY@tok@sd\endcsname{\let\PY@it=\textit\def\PY@tc##1{\textcolor[rgb]{0.73,0.13,0.13}{##1}}}
\expandafter\def\csname PY@tok@vg\endcsname{\def\PY@tc##1{\textcolor[rgb]{0.10,0.09,0.49}{##1}}}
\expandafter\def\csname PY@tok@gd\endcsname{\def\PY@tc##1{\textcolor[rgb]{0.63,0.00,0.00}{##1}}}
\expandafter\def\csname PY@tok@sh\endcsname{\def\PY@tc##1{\textcolor[rgb]{0.73,0.13,0.13}{##1}}}
\expandafter\def\csname PY@tok@kd\endcsname{\let\PY@bf=\textbf\def\PY@tc##1{\textcolor[rgb]{0.00,0.50,0.00}{##1}}}
\expandafter\def\csname PY@tok@gh\endcsname{\let\PY@bf=\textbf\def\PY@tc##1{\textcolor[rgb]{0.00,0.00,0.50}{##1}}}
\expandafter\def\csname PY@tok@mi\endcsname{\def\PY@tc##1{\textcolor[rgb]{0.40,0.40,0.40}{##1}}}
\expandafter\def\csname PY@tok@s2\endcsname{\def\PY@tc##1{\textcolor[rgb]{0.73,0.13,0.13}{##1}}}
\expandafter\def\csname PY@tok@mh\endcsname{\def\PY@tc##1{\textcolor[rgb]{0.40,0.40,0.40}{##1}}}
\expandafter\def\csname PY@tok@w\endcsname{\def\PY@tc##1{\textcolor[rgb]{0.73,0.73,0.73}{##1}}}
\expandafter\def\csname PY@tok@mb\endcsname{\def\PY@tc##1{\textcolor[rgb]{0.40,0.40,0.40}{##1}}}
\expandafter\def\csname PY@tok@nd\endcsname{\def\PY@tc##1{\textcolor[rgb]{0.67,0.13,1.00}{##1}}}
\expandafter\def\csname PY@tok@gp\endcsname{\let\PY@bf=\textbf\def\PY@tc##1{\textcolor[rgb]{0.00,0.00,0.50}{##1}}}

\def\PYZbs{\char`\\}
\def\PYZus{\char`\_}
\def\PYZob{\char`\{}
\def\PYZcb{\char`\}}
\def\PYZca{\char`\^}
\def\PYZam{\char`\&}
\def\PYZlt{\char`\<}
\def\PYZgt{\char`\>}
\def\PYZsh{\char`\#}
\def\PYZpc{\char`\%}
\def\PYZdl{\char`\$}
\def\PYZhy{\char`\-}
\def\PYZsq{\char`\'}
\def\PYZdq{\char`\"}
\def\PYZti{\char`\~}
% for compatibility with earlier versions
\def\PYZat{@}
\def\PYZlb{[}
\def\PYZrb{]}
\makeatother


    % Exact colors from NB
    \definecolor{incolor}{rgb}{0.0, 0.0, 0.5}
    \definecolor{outcolor}{rgb}{0.545, 0.0, 0.0}



    
    % Prevent overflowing lines due to hard-to-break entities
    \sloppy 
    % Setup hyperref package
    \hypersetup{
      breaklinks=true,  % so long urls are correctly broken across lines
      colorlinks=true,
      urlcolor=urlcolor,
      linkcolor=linkcolor,
      citecolor=citecolor,
      }
    % Slightly bigger margins than the latex defaults
    
    \geometry{verbose,tmargin=1in,bmargin=1in,lmargin=1in,rmargin=1in}
    
    

    \begin{document}
    
    
    \maketitle
    
    

    
    \section{Utilisation des listes sous forme stacks
(piles)}\label{utilisation-des-listes-sous-forme-stacks-piles}

Les méthodes de liste rendent très facile d'utiliser une liste comme une
pile, où le dernier élément ajouté est le premier élément récupéré
(«dernier entré, premier sorti»). Pour ajouter un élément au sommet de
la pile, utilisez append (). Pour récupérer un élément du haut de la
pile, utilisez pop () sans index explicite. Par exemple:

    \begin{Verbatim}[commandchars=\\\{\}]
{\color{incolor}In [{\color{incolor}70}]:} \PY{o}{\PYZgt{}\PYZgt{}}\PY{o}{\PYZgt{}} \PY{n}{stack} \PY{o}{=} \PY{p}{[}\PY{l+m+mi}{3}\PY{p}{,} \PY{l+m+mi}{4}\PY{p}{,} \PY{l+m+mi}{5}\PY{p}{]}
         \PY{o}{\PYZgt{}\PYZgt{}}\PY{o}{\PYZgt{}} \PY{n}{stack}\PY{o}{.}\PY{n}{append}\PY{p}{(}\PY{l+m+mi}{6}\PY{p}{)}
         \PY{o}{\PYZgt{}\PYZgt{}}\PY{o}{\PYZgt{}} \PY{n}{stack}\PY{o}{.}\PY{n}{append}\PY{p}{(}\PY{l+m+mi}{7}\PY{p}{)}
         \PY{o}{\PYZgt{}\PYZgt{}}\PY{o}{\PYZgt{}} \PY{n}{stack}
         \PY{p}{[}\PY{l+m+mi}{3}\PY{p}{,} \PY{l+m+mi}{4}\PY{p}{,} \PY{l+m+mi}{5}\PY{p}{,} \PY{l+m+mi}{6}\PY{p}{,} \PY{l+m+mi}{7}\PY{p}{]}
         \PY{o}{\PYZgt{}\PYZgt{}}\PY{o}{\PYZgt{}} \PY{n}{stack}\PY{o}{.}\PY{n}{pop}\PY{p}{(}\PY{p}{)}
         \PY{l+m+mi}{7}
         \PY{o}{\PYZgt{}\PYZgt{}}\PY{o}{\PYZgt{}} \PY{n}{stack}
         \PY{p}{[}\PY{l+m+mi}{3}\PY{p}{,} \PY{l+m+mi}{4}\PY{p}{,} \PY{l+m+mi}{5}\PY{p}{,} \PY{l+m+mi}{6}\PY{p}{]}
         \PY{o}{\PYZgt{}\PYZgt{}}\PY{o}{\PYZgt{}} \PY{n}{stack}\PY{o}{.}\PY{n}{pop}\PY{p}{(}\PY{p}{)}
         \PY{l+m+mi}{6}
         \PY{o}{\PYZgt{}\PYZgt{}}\PY{o}{\PYZgt{}} \PY{n}{stack}\PY{o}{.}\PY{n}{pop}\PY{p}{(}\PY{p}{)}
         \PY{l+m+mi}{5}
         \PY{o}{\PYZgt{}\PYZgt{}}\PY{o}{\PYZgt{}} \PY{n}{stack}
\end{Verbatim}

            \begin{Verbatim}[commandchars=\\\{\}]
{\color{outcolor}Out[{\color{outcolor}70}]:} [3, 4]
\end{Verbatim}
        
    \section{\texorpdfstring{Utilisation des listes en tant que ``queues''
(files
d'attente)}{Utilisation des listes en tant que queues (files d'attente)}}\label{utilisation-des-listes-en-tant-que-queues-files-dattente}

Il est également possible d'utiliser une liste comme une file d'attente,
où le premier élément ajouté est le premier élément récupéré («premier
entré, premier sorti»); Cependant, les listes ne sont pas efficaces à
cette fin. Alors que les appends et pops de la fin de la liste sont
rapides, faire des appends ou pops depuis le début d'une liste est lent
(car tous les autres éléments doivent être décalés d'un seul coup).

Pour implémenter une file d'attente, utilisez ``collections.deque'' qui
a été conçu pour avoir des appends rapides et pops des deux extrémités.
Par exemple:

    \begin{Verbatim}[commandchars=\\\{\}]
{\color{incolor}In [{\color{incolor}71}]:} \PY{o}{\PYZgt{}\PYZgt{}}\PY{o}{\PYZgt{}} \PY{k+kn}{from} \PY{n+nn}{collections} \PY{k}{import} \PY{n}{deque}
         \PY{o}{\PYZgt{}\PYZgt{}}\PY{o}{\PYZgt{}} \PY{n}{queue} \PY{o}{=} \PY{n}{deque}\PY{p}{(}\PY{p}{[}\PY{l+s+s2}{\PYZdq{}}\PY{l+s+s2}{Eric}\PY{l+s+s2}{\PYZdq{}}\PY{p}{,} \PY{l+s+s2}{\PYZdq{}}\PY{l+s+s2}{John}\PY{l+s+s2}{\PYZdq{}}\PY{p}{,} \PY{l+s+s2}{\PYZdq{}}\PY{l+s+s2}{Michael}\PY{l+s+s2}{\PYZdq{}}\PY{p}{]}\PY{p}{)}
         \PY{o}{\PYZgt{}\PYZgt{}}\PY{o}{\PYZgt{}} \PY{n}{queue}\PY{o}{.}\PY{n}{append}\PY{p}{(}\PY{l+s+s2}{\PYZdq{}}\PY{l+s+s2}{Terry}\PY{l+s+s2}{\PYZdq{}}\PY{p}{)}           \PY{c+c1}{\PYZsh{} Terry arrive}
         \PY{o}{\PYZgt{}\PYZgt{}}\PY{o}{\PYZgt{}} \PY{n}{queue}\PY{o}{.}\PY{n}{append}\PY{p}{(}\PY{l+s+s2}{\PYZdq{}}\PY{l+s+s2}{Graham}\PY{l+s+s2}{\PYZdq{}}\PY{p}{)}          \PY{c+c1}{\PYZsh{} Graham arrive}
         \PY{o}{\PYZgt{}\PYZgt{}}\PY{o}{\PYZgt{}} \PY{n}{queue}\PY{o}{.}\PY{n}{popleft}\PY{p}{(}\PY{p}{)}                 \PY{c+c1}{\PYZsh{} Le premier qui est arrivé part}
         \PY{l+s+s1}{\PYZsq{}}\PY{l+s+s1}{Eric}\PY{l+s+s1}{\PYZsq{}}
         \PY{o}{\PYZgt{}\PYZgt{}}\PY{o}{\PYZgt{}} \PY{n}{queue}\PY{o}{.}\PY{n}{popleft}\PY{p}{(}\PY{p}{)}                 \PY{c+c1}{\PYZsh{} Le deuxième qui est arrivé part}
         \PY{l+s+s1}{\PYZsq{}}\PY{l+s+s1}{John}\PY{l+s+s1}{\PYZsq{}}
         \PY{o}{\PYZgt{}\PYZgt{}}\PY{o}{\PYZgt{}} \PY{n}{queue}                           \PY{c+c1}{\PYZsh{} Renvoie la file d\PYZsq{}attente par ordre des arrivées}
         \PY{n}{deque}\PY{p}{(}\PY{p}{[}\PY{l+s+s1}{\PYZsq{}}\PY{l+s+s1}{Michael}\PY{l+s+s1}{\PYZsq{}}\PY{p}{,} \PY{l+s+s1}{\PYZsq{}}\PY{l+s+s1}{Terry}\PY{l+s+s1}{\PYZsq{}}\PY{p}{,} \PY{l+s+s1}{\PYZsq{}}\PY{l+s+s1}{Graham}\PY{l+s+s1}{\PYZsq{}}\PY{p}{]}\PY{p}{)}
\end{Verbatim}

            \begin{Verbatim}[commandchars=\\\{\}]
{\color{outcolor}Out[{\color{outcolor}71}]:} deque(['Michael', 'Terry', 'Graham'])
\end{Verbatim}
        
    \section{Les listes de
compréhensions}\label{les-listes-de-compruxe9hensions}

Les listes compréhensions fournissent un moyen concis de créer des
listes. Les applications courantes sont de faire de nouvelles listes où
chaque élément est le résultat de certaines opérations appliquées à
chaque membre d'une autre séquence ou itérables, ou de créer une
sous-séquence de ces éléments qui satisfont une certaine condition.

Par exemple, supposons que nous voulons créer une liste de carrés, comme
suit:

    \begin{Verbatim}[commandchars=\\\{\}]
{\color{incolor}In [{\color{incolor}72}]:} \PY{o}{\PYZgt{}\PYZgt{}}\PY{o}{\PYZgt{}} \PY{n}{squares} \PY{o}{=} \PY{p}{[}\PY{p}{]}
         \PY{o}{\PYZgt{}\PYZgt{}}\PY{o}{\PYZgt{}} \PY{k}{for} \PY{n}{x} \PY{o+ow}{in} \PY{n+nb}{range}\PY{p}{(}\PY{l+m+mi}{10}\PY{p}{)}\PY{p}{:}
         \PY{o}{.}\PY{o}{.}\PY{o}{.}     \PY{n}{squares}\PY{o}{.}\PY{n}{append}\PY{p}{(}\PY{n}{x}\PY{o}{*}\PY{o}{*}\PY{l+m+mi}{2}\PY{p}{)}
         \PY{o}{.}\PY{o}{.}\PY{o}{.}
         \PY{o}{\PYZgt{}\PYZgt{}}\PY{o}{\PYZgt{}} \PY{n}{squares}
\end{Verbatim}

            \begin{Verbatim}[commandchars=\\\{\}]
{\color{outcolor}Out[{\color{outcolor}72}]:} [0, 1, 4, 9, 16, 25, 36, 49, 64, 81]
\end{Verbatim}
        
    On peut obtenir le même résultat avec :

    \begin{Verbatim}[commandchars=\\\{\}]
{\color{incolor}In [{\color{incolor}73}]:} \PY{o}{\PYZgt{}\PYZgt{}}\PY{o}{\PYZgt{}} \PY{n}{squares} \PY{o}{=} \PY{p}{[}\PY{n}{x}\PY{o}{*}\PY{o}{*}\PY{l+m+mi}{2} \PY{k}{for} \PY{n}{x} \PY{o+ow}{in} \PY{n+nb}{range}\PY{p}{(}\PY{l+m+mi}{10}\PY{p}{)}\PY{p}{]}
         \PY{o}{\PYZgt{}\PYZgt{}}\PY{o}{\PYZgt{}} \PY{n}{squares}
\end{Verbatim}

            \begin{Verbatim}[commandchars=\\\{\}]
{\color{outcolor}Out[{\color{outcolor}73}]:} [0, 1, 4, 9, 16, 25, 36, 49, 64, 81]
\end{Verbatim}
        
    Une liste compréhension se compose de crochets contenant une expression
suivie par une clause for ou if. Le résultat sera une nouvelle liste
résultant de l'évaluation de l'expression dans le contexte des clauses
for et if qui la suivent. Par exemple, cette listcomp combine les
éléments de deux listes s'ils ne sont pas égaux:

    \begin{Verbatim}[commandchars=\\\{\}]
{\color{incolor}In [{\color{incolor}74}]:} \PY{o}{\PYZgt{}\PYZgt{}}\PY{o}{\PYZgt{}} \PY{p}{[}\PY{p}{(}\PY{n}{x}\PY{p}{,} \PY{n}{y}\PY{p}{)} \PY{k}{for} \PY{n}{x} \PY{o+ow}{in} \PY{p}{[}\PY{l+m+mi}{1}\PY{p}{,}\PY{l+m+mi}{2}\PY{p}{,}\PY{l+m+mi}{3}\PY{p}{]} \PY{k}{for} \PY{n}{y} \PY{o+ow}{in} \PY{p}{[}\PY{l+m+mi}{3}\PY{p}{,}\PY{l+m+mi}{1}\PY{p}{,}\PY{l+m+mi}{4}\PY{p}{]} \PY{k}{if} \PY{n}{x} \PY{o}{!=} \PY{n}{y}\PY{p}{]}
         \PY{p}{[}\PY{p}{(}\PY{l+m+mi}{1}\PY{p}{,} \PY{l+m+mi}{3}\PY{p}{)}\PY{p}{,} \PY{p}{(}\PY{l+m+mi}{1}\PY{p}{,} \PY{l+m+mi}{4}\PY{p}{)}\PY{p}{,} \PY{p}{(}\PY{l+m+mi}{2}\PY{p}{,} \PY{l+m+mi}{3}\PY{p}{)}\PY{p}{,} \PY{p}{(}\PY{l+m+mi}{2}\PY{p}{,} \PY{l+m+mi}{1}\PY{p}{)}\PY{p}{,} \PY{p}{(}\PY{l+m+mi}{2}\PY{p}{,} \PY{l+m+mi}{4}\PY{p}{)}\PY{p}{,} \PY{p}{(}\PY{l+m+mi}{3}\PY{p}{,} \PY{l+m+mi}{1}\PY{p}{)}\PY{p}{,} \PY{p}{(}\PY{l+m+mi}{3}\PY{p}{,} \PY{l+m+mi}{4}\PY{p}{)}\PY{p}{]}
\end{Verbatim}

            \begin{Verbatim}[commandchars=\\\{\}]
{\color{outcolor}Out[{\color{outcolor}74}]:} [(1, 3), (1, 4), (2, 3), (2, 1), (2, 4), (3, 1), (3, 4)]
\end{Verbatim}
        
    Qui est équivalent à :

    \begin{Verbatim}[commandchars=\\\{\}]
{\color{incolor}In [{\color{incolor}75}]:} \PY{o}{\PYZgt{}\PYZgt{}}\PY{o}{\PYZgt{}} \PY{n}{combs} \PY{o}{=} \PY{p}{[}\PY{p}{]}
         \PY{o}{\PYZgt{}\PYZgt{}}\PY{o}{\PYZgt{}} \PY{k}{for} \PY{n}{x} \PY{o+ow}{in} \PY{p}{[}\PY{l+m+mi}{1}\PY{p}{,}\PY{l+m+mi}{2}\PY{p}{,}\PY{l+m+mi}{3}\PY{p}{]}\PY{p}{:}
         \PY{o}{.}\PY{o}{.}\PY{o}{.}     \PY{k}{for} \PY{n}{y} \PY{o+ow}{in} \PY{p}{[}\PY{l+m+mi}{3}\PY{p}{,}\PY{l+m+mi}{1}\PY{p}{,}\PY{l+m+mi}{4}\PY{p}{]}\PY{p}{:}
         \PY{o}{.}\PY{o}{.}\PY{o}{.}         \PY{k}{if} \PY{n}{x} \PY{o}{!=} \PY{n}{y}\PY{p}{:}
         \PY{o}{.}\PY{o}{.}\PY{o}{.}             \PY{n}{combs}\PY{o}{.}\PY{n}{append}\PY{p}{(}\PY{p}{(}\PY{n}{x}\PY{p}{,} \PY{n}{y}\PY{p}{)}\PY{p}{)}
         \PY{o}{.}\PY{o}{.}\PY{o}{.}
         \PY{o}{\PYZgt{}\PYZgt{}}\PY{o}{\PYZgt{}} \PY{n}{combs}
\end{Verbatim}

            \begin{Verbatim}[commandchars=\\\{\}]
{\color{outcolor}Out[{\color{outcolor}75}]:} [(1, 3), (1, 4), (2, 3), (2, 1), (2, 4), (3, 1), (3, 4)]
\end{Verbatim}
        
    Notez comment l'ordre des instructions for et if est le même dans ces
deux extraits.

Si l'expression est un tuple (par exemple (x, y) dans l'exemple
précédent), elle doit être entre parenthèses.

    \begin{Verbatim}[commandchars=\\\{\}]
{\color{incolor}In [{\color{incolor}76}]:} \PY{o}{\PYZgt{}\PYZgt{}}\PY{o}{\PYZgt{}} \PY{n}{vec} \PY{o}{=} \PY{p}{[}\PY{o}{\PYZhy{}}\PY{l+m+mi}{4}\PY{p}{,} \PY{o}{\PYZhy{}}\PY{l+m+mi}{2}\PY{p}{,} \PY{l+m+mi}{0}\PY{p}{,} \PY{l+m+mi}{2}\PY{p}{,} \PY{l+m+mi}{4}\PY{p}{]}
         \PY{o}{\PYZgt{}\PYZgt{}}\PY{o}{\PYZgt{}} \PY{c+c1}{\PYZsh{} Créé une nouvelle liste avec des valeurs doublé}
         \PY{o}{\PYZgt{}\PYZgt{}}\PY{o}{\PYZgt{}} \PY{p}{[}\PY{n}{x}\PY{o}{*}\PY{l+m+mi}{2} \PY{k}{for} \PY{n}{x} \PY{o+ow}{in} \PY{n}{vec}\PY{p}{]}
         \PY{p}{[}\PY{o}{\PYZhy{}}\PY{l+m+mi}{8}\PY{p}{,} \PY{o}{\PYZhy{}}\PY{l+m+mi}{4}\PY{p}{,} \PY{l+m+mi}{0}\PY{p}{,} \PY{l+m+mi}{4}\PY{p}{,} \PY{l+m+mi}{8}\PY{p}{]}
         \PY{o}{\PYZgt{}\PYZgt{}}\PY{o}{\PYZgt{}} \PY{c+c1}{\PYZsh{} Filtre la liste pour exclure les nombres négatifs}
         \PY{o}{\PYZgt{}\PYZgt{}}\PY{o}{\PYZgt{}} \PY{p}{[}\PY{n}{x} \PY{k}{for} \PY{n}{x} \PY{o+ow}{in} \PY{n}{vec} \PY{k}{if} \PY{n}{x} \PY{o}{\PYZgt{}}\PY{o}{=} \PY{l+m+mi}{0}\PY{p}{]}
         \PY{p}{[}\PY{l+m+mi}{0}\PY{p}{,} \PY{l+m+mi}{2}\PY{p}{,} \PY{l+m+mi}{4}\PY{p}{]}
         \PY{o}{\PYZgt{}\PYZgt{}}\PY{o}{\PYZgt{}} \PY{c+c1}{\PYZsh{} Applique une fonction à tous les éléments}
         \PY{o}{\PYZgt{}\PYZgt{}}\PY{o}{\PYZgt{}} \PY{p}{[}\PY{n+nb}{abs}\PY{p}{(}\PY{n}{x}\PY{p}{)} \PY{k}{for} \PY{n}{x} \PY{o+ow}{in} \PY{n}{vec}\PY{p}{]}
         \PY{p}{[}\PY{l+m+mi}{4}\PY{p}{,} \PY{l+m+mi}{2}\PY{p}{,} \PY{l+m+mi}{0}\PY{p}{,} \PY{l+m+mi}{2}\PY{p}{,} \PY{l+m+mi}{4}\PY{p}{]}
         \PY{o}{\PYZgt{}\PYZgt{}}\PY{o}{\PYZgt{}} \PY{c+c1}{\PYZsh{} Appel la methode sur chaque éléments}
         \PY{o}{\PYZgt{}\PYZgt{}}\PY{o}{\PYZgt{}} \PY{n}{freshfruit} \PY{o}{=} \PY{p}{[}\PY{l+s+s1}{\PYZsq{}}\PY{l+s+s1}{  banana}\PY{l+s+s1}{\PYZsq{}}\PY{p}{,} \PY{l+s+s1}{\PYZsq{}}\PY{l+s+s1}{  loganberry }\PY{l+s+s1}{\PYZsq{}}\PY{p}{,} \PY{l+s+s1}{\PYZsq{}}\PY{l+s+s1}{passion fruit  }\PY{l+s+s1}{\PYZsq{}}\PY{p}{]}
         \PY{o}{\PYZgt{}\PYZgt{}}\PY{o}{\PYZgt{}} \PY{p}{[}\PY{n}{weapon}\PY{o}{.}\PY{n}{strip}\PY{p}{(}\PY{p}{)} \PY{k}{for} \PY{n}{weapon} \PY{o+ow}{in} \PY{n}{freshfruit}\PY{p}{]}
         \PY{p}{[}\PY{l+s+s1}{\PYZsq{}}\PY{l+s+s1}{banana}\PY{l+s+s1}{\PYZsq{}}\PY{p}{,} \PY{l+s+s1}{\PYZsq{}}\PY{l+s+s1}{loganberry}\PY{l+s+s1}{\PYZsq{}}\PY{p}{,} \PY{l+s+s1}{\PYZsq{}}\PY{l+s+s1}{passion fruit}\PY{l+s+s1}{\PYZsq{}}\PY{p}{]}
         \PY{o}{\PYZgt{}\PYZgt{}}\PY{o}{\PYZgt{}} \PY{c+c1}{\PYZsh{} Créé une liste de 2\PYZhy{}tuples comme (number, square)}
         \PY{o}{\PYZgt{}\PYZgt{}}\PY{o}{\PYZgt{}} \PY{p}{[}\PY{p}{(}\PY{n}{x}\PY{p}{,} \PY{n}{x}\PY{o}{*}\PY{o}{*}\PY{l+m+mi}{2}\PY{p}{)} \PY{k}{for} \PY{n}{x} \PY{o+ow}{in} \PY{n+nb}{range}\PY{p}{(}\PY{l+m+mi}{6}\PY{p}{)}\PY{p}{]}
         \PY{p}{[}\PY{p}{(}\PY{l+m+mi}{0}\PY{p}{,} \PY{l+m+mi}{0}\PY{p}{)}\PY{p}{,} \PY{p}{(}\PY{l+m+mi}{1}\PY{p}{,} \PY{l+m+mi}{1}\PY{p}{)}\PY{p}{,} \PY{p}{(}\PY{l+m+mi}{2}\PY{p}{,} \PY{l+m+mi}{4}\PY{p}{)}\PY{p}{,} \PY{p}{(}\PY{l+m+mi}{3}\PY{p}{,} \PY{l+m+mi}{9}\PY{p}{)}\PY{p}{,} \PY{p}{(}\PY{l+m+mi}{4}\PY{p}{,} \PY{l+m+mi}{16}\PY{p}{)}\PY{p}{,} \PY{p}{(}\PY{l+m+mi}{5}\PY{p}{,} \PY{l+m+mi}{25}\PY{p}{)}\PY{p}{]}
         \PY{o}{\PYZgt{}\PYZgt{}}\PY{o}{\PYZgt{}} \PY{c+c1}{\PYZsh{} Le tuple doit être entre parenthèse, sinon une erreur est lévée}
         \PY{o}{\PYZgt{}\PYZgt{}}\PY{o}{\PYZgt{}} \PY{p}{[}\PY{n}{x}\PY{p}{,} \PY{n}{x}\PY{o}{*}\PY{o}{*}\PY{l+m+mi}{2} \PY{k}{for} \PY{n}{x} \PY{o+ow}{in} \PY{n+nb}{range}\PY{p}{(}\PY{l+m+mi}{6}\PY{p}{)}\PY{p}{]}
\end{Verbatim}

    \begin{Verbatim}[commandchars=\\\{\}]

          File "<ipython-input-76-25774bc1b033>", line 19
        [x, x**2 for x in range(6)]
                   \^{}
    SyntaxError: invalid syntax
    

    \end{Verbatim}

    \begin{Verbatim}[commandchars=\\\{\}]
{\color{incolor}In [{\color{incolor} }]:} \PY{o}{\PYZgt{}\PYZgt{}}\PY{o}{\PYZgt{}} \PY{c+c1}{\PYZsh{} Aplati une liste en utilisant un listcomp avec deux \PYZsq{}for\PYZsq{}}
        \PY{o}{\PYZgt{}\PYZgt{}}\PY{o}{\PYZgt{}} \PY{n}{vec} \PY{o}{=} \PY{p}{[}\PY{p}{[}\PY{l+m+mi}{1}\PY{p}{,}\PY{l+m+mi}{2}\PY{p}{,}\PY{l+m+mi}{3}\PY{p}{]}\PY{p}{,} \PY{p}{[}\PY{l+m+mi}{4}\PY{p}{,}\PY{l+m+mi}{5}\PY{p}{,}\PY{l+m+mi}{6}\PY{p}{]}\PY{p}{,} \PY{p}{[}\PY{l+m+mi}{7}\PY{p}{,}\PY{l+m+mi}{8}\PY{p}{,}\PY{l+m+mi}{9}\PY{p}{]}\PY{p}{]}
        \PY{o}{\PYZgt{}\PYZgt{}}\PY{o}{\PYZgt{}} \PY{p}{[}\PY{n}{num} \PY{k}{for} \PY{n}{elem} \PY{o+ow}{in} \PY{n}{vec} \PY{k}{for} \PY{n}{num} \PY{o+ow}{in} \PY{n}{elem}\PY{p}{]}
\end{Verbatim}

    Les listes compréhensions peuvent contenir des expressions complexes et
des fonctions imbriquées:

    \begin{Verbatim}[commandchars=\\\{\}]
{\color{incolor}In [{\color{incolor} }]:} \PY{o}{\PYZgt{}\PYZgt{}}\PY{o}{\PYZgt{}} \PY{k+kn}{from} \PY{n+nn}{math} \PY{k}{import} \PY{n}{pi}
        \PY{o}{\PYZgt{}\PYZgt{}}\PY{o}{\PYZgt{}} \PY{p}{[}\PY{n+nb}{str}\PY{p}{(}\PY{n+nb}{round}\PY{p}{(}\PY{n}{pi}\PY{p}{,} \PY{n}{i}\PY{p}{)}\PY{p}{)} \PY{k}{for} \PY{n}{i} \PY{o+ow}{in} \PY{n+nb}{range}\PY{p}{(}\PY{l+m+mi}{1}\PY{p}{,} \PY{l+m+mi}{6}\PY{p}{)}\PY{p}{]}
        \PY{p}{[}\PY{l+s+s1}{\PYZsq{}}\PY{l+s+s1}{3.1}\PY{l+s+s1}{\PYZsq{}}\PY{p}{,} \PY{l+s+s1}{\PYZsq{}}\PY{l+s+s1}{3.14}\PY{l+s+s1}{\PYZsq{}}\PY{p}{,} \PY{l+s+s1}{\PYZsq{}}\PY{l+s+s1}{3.142}\PY{l+s+s1}{\PYZsq{}}\PY{p}{,} \PY{l+s+s1}{\PYZsq{}}\PY{l+s+s1}{3.1416}\PY{l+s+s1}{\PYZsq{}}\PY{p}{,} \PY{l+s+s1}{\PYZsq{}}\PY{l+s+s1}{3.14159}\PY{l+s+s1}{\PYZsq{}}\PY{p}{]}
\end{Verbatim}

    \section{Les listes imbriquées de
compréhensions}\label{les-listes-imbriquuxe9es-de-compruxe9hensions}

L'expression initiale dans une liste de compréhension peut être
n'importe quelle expression arbitraire, y compris une autre liste de
compréhension.

Considérons l'exemple suivant d'une matrice 3x4 implémentée comme une
liste de 3 listes de longueur 4:

    \begin{Verbatim}[commandchars=\\\{\}]
{\color{incolor}In [{\color{incolor} }]:} \PY{o}{\PYZgt{}\PYZgt{}}\PY{o}{\PYZgt{}} \PY{n}{matrix} \PY{o}{=} \PY{p}{[}
        \PY{o}{.}\PY{o}{.}\PY{o}{.}     \PY{p}{[}\PY{l+m+mi}{1}\PY{p}{,} \PY{l+m+mi}{2}\PY{p}{,} \PY{l+m+mi}{3}\PY{p}{,} \PY{l+m+mi}{4}\PY{p}{]}\PY{p}{,}
        \PY{o}{.}\PY{o}{.}\PY{o}{.}     \PY{p}{[}\PY{l+m+mi}{5}\PY{p}{,} \PY{l+m+mi}{6}\PY{p}{,} \PY{l+m+mi}{7}\PY{p}{,} \PY{l+m+mi}{8}\PY{p}{]}\PY{p}{,}
        \PY{o}{.}\PY{o}{.}\PY{o}{.}     \PY{p}{[}\PY{l+m+mi}{9}\PY{p}{,} \PY{l+m+mi}{10}\PY{p}{,} \PY{l+m+mi}{11}\PY{p}{,} \PY{l+m+mi}{12}\PY{p}{]}\PY{p}{,}
        \PY{o}{.}\PY{o}{.}\PY{o}{.} \PY{p}{]}
\end{Verbatim}

    La liste suivante transposera les lignes et les colonnes

    \begin{Verbatim}[commandchars=\\\{\}]
{\color{incolor}In [{\color{incolor} }]:} \PY{o}{\PYZgt{}\PYZgt{}}\PY{o}{\PYZgt{}} \PY{p}{[}\PY{p}{[}\PY{n}{row}\PY{p}{[}\PY{n}{i}\PY{p}{]} \PY{k}{for} \PY{n}{row} \PY{o+ow}{in} \PY{n}{matrix}\PY{p}{]} \PY{k}{for} \PY{n}{i} \PY{o+ow}{in} \PY{n+nb}{range}\PY{p}{(}\PY{l+m+mi}{4}\PY{p}{)}\PY{p}{]}
\end{Verbatim}

    Comme nous l'avons vu dans la section précédente, le listcomp imbriqué
est évalué dans le contexte du for qui le suit, donc cet exemple est
équivalent à:

    \begin{Verbatim}[commandchars=\\\{\}]
{\color{incolor}In [{\color{incolor} }]:} \PY{o}{\PYZgt{}\PYZgt{}}\PY{o}{\PYZgt{}} \PY{n}{transposed} \PY{o}{=} \PY{p}{[}\PY{p}{]}
        \PY{o}{\PYZgt{}\PYZgt{}}\PY{o}{\PYZgt{}} \PY{k}{for} \PY{n}{i} \PY{o+ow}{in} \PY{n+nb}{range}\PY{p}{(}\PY{l+m+mi}{4}\PY{p}{)}\PY{p}{:}
        \PY{o}{.}\PY{o}{.}\PY{o}{.}     \PY{n}{transposed}\PY{o}{.}\PY{n}{append}\PY{p}{(}\PY{p}{[}\PY{n}{row}\PY{p}{[}\PY{n}{i}\PY{p}{]} \PY{k}{for} \PY{n}{row} \PY{o+ow}{in} \PY{n}{matrix}\PY{p}{]}\PY{p}{)}
        \PY{o}{.}\PY{o}{.}\PY{o}{.}
        \PY{o}{\PYZgt{}\PYZgt{}}\PY{o}{\PYZgt{}} \PY{n}{transposed}
\end{Verbatim}

    Qui, à son tour, est la même que:

    \begin{Verbatim}[commandchars=\\\{\}]
{\color{incolor}In [{\color{incolor} }]:} \PY{o}{\PYZgt{}\PYZgt{}}\PY{o}{\PYZgt{}} \PY{n}{transposed} \PY{o}{=} \PY{p}{[}\PY{p}{]}
        \PY{o}{\PYZgt{}\PYZgt{}}\PY{o}{\PYZgt{}} \PY{k}{for} \PY{n}{i} \PY{o+ow}{in} \PY{n+nb}{range}\PY{p}{(}\PY{l+m+mi}{4}\PY{p}{)}\PY{p}{:}
        \PY{o}{.}\PY{o}{.}\PY{o}{.}     \PY{c+c1}{\PYZsh{} Les 3 lignes suivantes implémentent une listcomp imbriqué}
        \PY{o}{.}\PY{o}{.}\PY{o}{.}     \PY{n}{transposed\PYZus{}row} \PY{o}{=} \PY{p}{[}\PY{p}{]}
        \PY{o}{.}\PY{o}{.}\PY{o}{.}     \PY{k}{for} \PY{n}{row} \PY{o+ow}{in} \PY{n}{matrix}\PY{p}{:}
        \PY{o}{.}\PY{o}{.}\PY{o}{.}         \PY{n}{transposed\PYZus{}row}\PY{o}{.}\PY{n}{append}\PY{p}{(}\PY{n}{row}\PY{p}{[}\PY{n}{i}\PY{p}{]}\PY{p}{)}
        \PY{o}{.}\PY{o}{.}\PY{o}{.}     \PY{n}{transposed}\PY{o}{.}\PY{n}{append}\PY{p}{(}\PY{n}{transposed\PYZus{}row}\PY{p}{)}
        \PY{o}{.}\PY{o}{.}\PY{o}{.}
        \PY{o}{\PYZgt{}\PYZgt{}}\PY{o}{\PYZgt{}} \PY{n}{transposed}
\end{Verbatim}

    Dans le monde réel, on préférera les fonctions intégrées aux
instructions de flux complexes. La fonction zip () ferait un excellent
travail pour ce cas d'utilisation:

    \begin{Verbatim}[commandchars=\\\{\}]
{\color{incolor}In [{\color{incolor} }]:} \PY{o}{\PYZgt{}\PYZgt{}}\PY{o}{\PYZgt{}} \PY{n+nb}{list}\PY{p}{(}\PY{n+nb}{zip}\PY{p}{(}\PY{o}{*}\PY{n}{matrix}\PY{p}{)}\PY{p}{)}
\end{Verbatim}

    \section{Tuples et séquences}\label{tuples-et-suxe9quences}

Nous avons vu que les listes et les chaînes ont de nombreuses propriétés
communes, telles que l'indexation et les opérations de découpage. Il
s'agit de deux exemples de types de données de séquence (voir Types de
séquences - liste, tuple, plage). Puisque Python est un langage
évolutif, d'autres types de données de séquence peuvent être ajoutés. Il
existe également un autre type de données de séquence standard: le
tuple.

Un tuple se compose d'un certain nombre de valeurs séparées par des
virgules, par exemple

    \begin{Verbatim}[commandchars=\\\{\}]
{\color{incolor}In [{\color{incolor} }]:} \PY{o}{\PYZgt{}\PYZgt{}}\PY{o}{\PYZgt{}} \PY{n}{t} \PY{o}{=} \PY{l+m+mi}{12345}\PY{p}{,} \PY{l+m+mi}{54321}\PY{p}{,} \PY{l+s+s1}{\PYZsq{}}\PY{l+s+s1}{hello!}\PY{l+s+s1}{\PYZsq{}}
        \PY{o}{\PYZgt{}\PYZgt{}}\PY{o}{\PYZgt{}} \PY{n}{t}\PY{p}{[}\PY{l+m+mi}{0}\PY{p}{]}
\end{Verbatim}

    \begin{Verbatim}[commandchars=\\\{\}]
{\color{incolor}In [{\color{incolor} }]:} \PY{o}{\PYZgt{}\PYZgt{}}\PY{o}{\PYZgt{}} \PY{n}{t}
\end{Verbatim}

    \begin{Verbatim}[commandchars=\\\{\}]
{\color{incolor}In [{\color{incolor} }]:} \PY{o}{\PYZgt{}\PYZgt{}}\PY{o}{\PYZgt{}} \PY{c+c1}{\PYZsh{} Les tuples peuvent être imbriqué:}
        \PY{o}{.}\PY{o}{.}\PY{o}{.} \PY{n}{u} \PY{o}{=} \PY{n}{t}\PY{p}{,} \PY{p}{(}\PY{l+m+mi}{1}\PY{p}{,} \PY{l+m+mi}{2}\PY{p}{,} \PY{l+m+mi}{3}\PY{p}{,} \PY{l+m+mi}{4}\PY{p}{,} \PY{l+m+mi}{5}\PY{p}{)}
        \PY{o}{\PYZgt{}\PYZgt{}}\PY{o}{\PYZgt{}} \PY{n}{u}
\end{Verbatim}

    \begin{Verbatim}[commandchars=\\\{\}]
{\color{incolor}In [{\color{incolor} }]:} \PY{o}{\PYZgt{}\PYZgt{}}\PY{o}{\PYZgt{}} \PY{c+c1}{\PYZsh{} Les tuple sont immuables:}
        \PY{o}{.}\PY{o}{.}\PY{o}{.} \PY{n}{t}\PY{p}{[}\PY{l+m+mi}{0}\PY{p}{]} \PY{o}{=} \PY{l+m+mi}{88888}
\end{Verbatim}

    \begin{Verbatim}[commandchars=\\\{\}]
{\color{incolor}In [{\color{incolor} }]:} \PY{o}{\PYZgt{}\PYZgt{}}\PY{o}{\PYZgt{}} \PY{c+c1}{\PYZsh{} Mais ils peuvent contenir des objets mutable:}
        \PY{o}{.}\PY{o}{.}\PY{o}{.} \PY{n}{v} \PY{o}{=} \PY{p}{(}\PY{p}{[}\PY{l+m+mi}{1}\PY{p}{,} \PY{l+m+mi}{2}\PY{p}{,} \PY{l+m+mi}{3}\PY{p}{]}\PY{p}{,} \PY{p}{[}\PY{l+m+mi}{3}\PY{p}{,} \PY{l+m+mi}{2}\PY{p}{,} \PY{l+m+mi}{1}\PY{p}{]}\PY{p}{)}
        \PY{o}{\PYZgt{}\PYZgt{}}\PY{o}{\PYZgt{}} \PY{n}{v}
\end{Verbatim}

    Comme vous le voyez, les tuples de sortie sont toujours entre
parenthèses, de sorte que les tuples imbriqués soit interprétés
correctement. Ils peuvent être entrés avec ou sans parenthèses
entourant, bien que souvent des parenthèses soient nécessaires de toute
façon (si le tuple fait partie d'une expression plus grande). Il n'est
pas possible d'attribuer aux éléments individuels d'un tuple, mais il
est possible de créer des tuples qui contiennent des objets mutables,
tels que des listes.

Bien que les tuples peuvent sembler similaires aux listes, ils sont
souvent utilisés dans des situations différentes et à des fins
différentes. Les tuples sont immuables et contiennent généralement une
séquence hétérogène d'éléments qui sont accessibles via le déballage
(voir plus loin dans cette section) ou l'indexation (ou même par
attribut dans le cas de namedpleples). Les listes sont mutables, et
leurs éléments sont généralement homogènes et sont accessibles par
itération sur la liste.

Un problème particulier est la construction de tuples contenant 0 ou 1
éléments: la syntaxe a quelques bizarreries supplémentaires pour les
accommoder. Les tuples vides sont construits par une paire vide de
parenthèses. Un tuple avec un élément est construit en suivant une
valeur avec une virgule (il ne suffit pas d'entourer une seule valeur
entre parenthèses). Moche, mais efficace. Par exemple:

    \begin{Verbatim}[commandchars=\\\{\}]
{\color{incolor}In [{\color{incolor} }]:} \PY{o}{\PYZgt{}\PYZgt{}}\PY{o}{\PYZgt{}} \PY{n}{empty} \PY{o}{=} \PY{p}{(}\PY{p}{)}
        \PY{o}{\PYZgt{}\PYZgt{}}\PY{o}{\PYZgt{}} \PY{n}{singleton} \PY{o}{=} \PY{l+s+s1}{\PYZsq{}}\PY{l+s+s1}{hello}\PY{l+s+s1}{\PYZsq{}}\PY{p}{,}    \PY{c+c1}{\PYZsh{} \PYZlt{}\PYZhy{}\PYZhy{} notez la virgule}
        \PY{o}{\PYZgt{}\PYZgt{}}\PY{o}{\PYZgt{}} \PY{n+nb}{len}\PY{p}{(}\PY{n}{empty}\PY{p}{)}
\end{Verbatim}

    \begin{Verbatim}[commandchars=\\\{\}]
{\color{incolor}In [{\color{incolor} }]:} \PY{n+nb}{len}\PY{p}{(}\PY{n}{singleton}\PY{p}{)}
\end{Verbatim}

    \begin{Verbatim}[commandchars=\\\{\}]
{\color{incolor}In [{\color{incolor} }]:} \PY{o}{\PYZgt{}\PYZgt{}}\PY{o}{\PYZgt{}} \PY{n}{singleton}
\end{Verbatim}

    La déclaration t = 12345, 54321, `bonjour!' Est un exemple d'emballage
de tuples: les valeurs 12345, 54321 et `hello!' Sont emballés ensemble
dans un tuple. L'inversion est également possible:

    \begin{Verbatim}[commandchars=\\\{\}]
{\color{incolor}In [{\color{incolor} }]:} \PY{o}{\PYZgt{}\PYZgt{}}\PY{o}{\PYZgt{}} \PY{n}{x}\PY{p}{,} \PY{n}{y}\PY{p}{,} \PY{n}{z} \PY{o}{=} \PY{n}{t}
\end{Verbatim}

    C'est ce qu'on appelle, de façon appropriée, séquence de déballage et
fonctionne pour n'importe quelle séquence sur le côté droit. Le
déballage des séquences exige qu'il y ait autant de variables sur le
côté gauche du signe égal que d'éléments dans la séquence. Notez que
l'assignation multiple est vraiment juste une combinaison de l'emballage
de tuple et déballage de séquence.

    \section{Les sets}\label{les-sets}

Python inclut également un type de données pour les ensembles. Un
ensemble est une collection non ordonnée sans éléments dupliqués. Les
utilisations de base comprennent les tests d'adhésion et l'élimination
des entrées en double. Les objets définis prennent également en charge
des opérations mathématiques telles que l'union, l'intersection, la
différence et la différence symétrique.

Les accolades ou la fonction set () peuvent être utilisées pour créer
des ensembles. Remarque: pour créer un jeu vide, vous devez utiliser set
(), not \{\}; Ce dernier crée un dictionnaire vide, une structure de
données dont nous discuterons dans la section suivante.

Voici une brève démonstration:

    \begin{Verbatim}[commandchars=\\\{\}]
{\color{incolor}In [{\color{incolor} }]:} \PY{o}{\PYZgt{}\PYZgt{}}\PY{o}{\PYZgt{}} \PY{n}{basket} \PY{o}{=} \PY{p}{\PYZob{}}\PY{l+s+s1}{\PYZsq{}}\PY{l+s+s1}{apple}\PY{l+s+s1}{\PYZsq{}}\PY{p}{,} \PY{l+s+s1}{\PYZsq{}}\PY{l+s+s1}{orange}\PY{l+s+s1}{\PYZsq{}}\PY{p}{,} \PY{l+s+s1}{\PYZsq{}}\PY{l+s+s1}{apple}\PY{l+s+s1}{\PYZsq{}}\PY{p}{,} \PY{l+s+s1}{\PYZsq{}}\PY{l+s+s1}{pear}\PY{l+s+s1}{\PYZsq{}}\PY{p}{,} \PY{l+s+s1}{\PYZsq{}}\PY{l+s+s1}{orange}\PY{l+s+s1}{\PYZsq{}}\PY{p}{,} \PY{l+s+s1}{\PYZsq{}}\PY{l+s+s1}{banana}\PY{l+s+s1}{\PYZsq{}}\PY{p}{\PYZcb{}}
        \PY{o}{\PYZgt{}\PYZgt{}}\PY{o}{\PYZgt{}} \PY{n+nb}{print}\PY{p}{(}\PY{n}{basket}\PY{p}{)}  
\end{Verbatim}

    \begin{Verbatim}[commandchars=\\\{\}]
{\color{incolor}In [{\color{incolor} }]:} \PY{o}{\PYZgt{}\PYZgt{}}\PY{o}{\PYZgt{}} \PY{l+s+s1}{\PYZsq{}}\PY{l+s+s1}{orange}\PY{l+s+s1}{\PYZsq{}} \PY{o+ow}{in} \PY{n}{basket} 
\end{Verbatim}

    \begin{Verbatim}[commandchars=\\\{\}]
{\color{incolor}In [{\color{incolor} }]:} \PY{o}{\PYZgt{}\PYZgt{}}\PY{o}{\PYZgt{}} \PY{l+s+s1}{\PYZsq{}}\PY{l+s+s1}{crabgrass}\PY{l+s+s1}{\PYZsq{}} \PY{o+ow}{in} \PY{n}{basket}
\end{Verbatim}

    \begin{Verbatim}[commandchars=\\\{\}]
{\color{incolor}In [{\color{incolor} }]:} \PY{o}{\PYZgt{}\PYZgt{}}\PY{o}{\PYZgt{}} \PY{c+c1}{\PYZsh{} Démontre des opérations sur des lettres uniques à partir de deux mots}
        \PY{o}{.}\PY{o}{.}\PY{o}{.}
        \PY{o}{\PYZgt{}\PYZgt{}}\PY{o}{\PYZgt{}} \PY{n}{a} \PY{o}{=} \PY{n+nb}{set}\PY{p}{(}\PY{l+s+s1}{\PYZsq{}}\PY{l+s+s1}{abracadabra}\PY{l+s+s1}{\PYZsq{}}\PY{p}{)}
        \PY{o}{\PYZgt{}\PYZgt{}}\PY{o}{\PYZgt{}} \PY{n}{b} \PY{o}{=} \PY{n+nb}{set}\PY{p}{(}\PY{l+s+s1}{\PYZsq{}}\PY{l+s+s1}{alacazam}\PY{l+s+s1}{\PYZsq{}}\PY{p}{)}
        \PY{o}{\PYZgt{}\PYZgt{}}\PY{o}{\PYZgt{}} \PY{n}{a}      
\end{Verbatim}

    \begin{Verbatim}[commandchars=\\\{\}]
{\color{incolor}In [{\color{incolor} }]:} \PY{o}{\PYZgt{}\PYZgt{}}\PY{o}{\PYZgt{}} \PY{n}{a} \PY{o}{\PYZhy{}} \PY{n}{b}                              \PY{c+c1}{\PYZsh{} Les lettres dans a mais pas dans b}
\end{Verbatim}

    \begin{Verbatim}[commandchars=\\\{\}]
{\color{incolor}In [{\color{incolor} }]:} \PY{o}{\PYZgt{}\PYZgt{}}\PY{o}{\PYZgt{}} \PY{n}{a} \PY{o}{|} \PY{n}{b}                              \PY{c+c1}{\PYZsh{} lettres ni dans a ou b}
\end{Verbatim}

    \begin{Verbatim}[commandchars=\\\{\}]
{\color{incolor}In [{\color{incolor} }]:} \PY{o}{\PYZgt{}\PYZgt{}}\PY{o}{\PYZgt{}} \PY{n}{a} \PY{o}{\PYZam{}} \PY{n}{b}                              \PY{c+c1}{\PYZsh{} lettres dans a et b}
\end{Verbatim}

    \begin{Verbatim}[commandchars=\\\{\}]
{\color{incolor}In [{\color{incolor} }]:} \PY{o}{\PYZgt{}\PYZgt{}}\PY{o}{\PYZgt{}} \PY{n}{a} \PY{o}{\PYZca{}} \PY{n}{b}                              \PY{c+c1}{\PYZsh{} lettres dans a ou b mais pas dans les deux}
\end{Verbatim}

    De la même manière que pour la liste des compréhensions, les ensembles
de compréhensions sont également pris en charge:

    \begin{Verbatim}[commandchars=\\\{\}]
{\color{incolor}In [{\color{incolor} }]:} \PY{o}{\PYZgt{}\PYZgt{}}\PY{o}{\PYZgt{}} \PY{n}{a} \PY{o}{=} \PY{p}{\PYZob{}}\PY{n}{x} \PY{k}{for} \PY{n}{x} \PY{o+ow}{in} \PY{l+s+s1}{\PYZsq{}}\PY{l+s+s1}{abracadabra}\PY{l+s+s1}{\PYZsq{}} \PY{k}{if} \PY{n}{x} \PY{o+ow}{not} \PY{o+ow}{in} \PY{l+s+s1}{\PYZsq{}}\PY{l+s+s1}{abc}\PY{l+s+s1}{\PYZsq{}}\PY{p}{\PYZcb{}}
        \PY{o}{\PYZgt{}\PYZgt{}}\PY{o}{\PYZgt{}} \PY{n}{a}
\end{Verbatim}

    \section{Les dictionnaires}\label{les-dictionnaires}

Un autre type de données utile intégré à Python est le dictionnaire
(voir Types de mappage - dict). Les dictionnaires sont parfois trouvés
dans d'autres langues comme «mémoires associatives» ou «tableaux
associatifs». Contrairement aux séquences, qui sont indexées par une
plage de nombres, les dictionnaires sont indexés par des clés, qui
peuvent être n'importe quel type immuable; Les chaînes et les nombres
peuvent toujours être des clés. Les tuples peuvent être utilisés comme
des clés si elles ne contiennent que des chaînes, des nombres ou des
tuples; Si un tuple contient un objet mutable directement ou
indirectement, il ne peut pas être utilisé comme clé. Vous ne pouvez pas
utiliser les listes comme clés, car les listes peuvent être modifiées en
place à l'aide d'affectations d'index, d'affectations de tranches ou de
méthodes comme append () et extend ().

Il est préférable de penser à un dictionnaire comme un ensemble non
désordonné de paires clé: valeur, avec l'exigence que les clés sont
uniques (dans un dictionnaire). Une paire d'accolades crée un
dictionnaire vide: \{\}. Placer une liste séparée par des virgules de
paires clé: valeur dans les accolades ajoute des paires clé: valeur au
dictionnaire. C'est aussi la façon dont les dictionnaires sont écrits à
la sortie.

Les principales opérations sur un dictionnaire sont stocker une valeur
avec une certaine clé et d'extraire la valeur donnée la clé. Il est
également possible de supprimer une paire clé: valeur avec del. Si vous
stockez en utilisant une clé déjà utilisée, l'ancienne valeur associée à
cette clé est oubliée. Il s'agit d'une erreur pour extraire une valeur à
l'aide d'une clé inexistante.

La liste d'exécution (d.keys ()) sur un dictionnaire retourne une liste
de toutes les clés utilisées dans le dictionnaire, dans un ordre
arbitraire (si vous voulez trier, il suffit d'utiliser trié (d.keys
())). {[}2{]} Pour vérifier si une seule clé est dans le dictionnaire,
utilisez le mot clé ``in''.

Voici un petit exemple utilisant un dictionnaire:

    \begin{Verbatim}[commandchars=\\\{\}]
{\color{incolor}In [{\color{incolor} }]:} \PY{o}{\PYZgt{}\PYZgt{}}\PY{o}{\PYZgt{}} \PY{n}{tel} \PY{o}{=} \PY{p}{\PYZob{}}\PY{l+s+s1}{\PYZsq{}}\PY{l+s+s1}{jack}\PY{l+s+s1}{\PYZsq{}}\PY{p}{:} \PY{l+m+mi}{4098}\PY{p}{,} \PY{l+s+s1}{\PYZsq{}}\PY{l+s+s1}{sape}\PY{l+s+s1}{\PYZsq{}}\PY{p}{:} \PY{l+m+mi}{4139}\PY{p}{\PYZcb{}}
        \PY{o}{\PYZgt{}\PYZgt{}}\PY{o}{\PYZgt{}} \PY{n}{tel}\PY{p}{[}\PY{l+s+s1}{\PYZsq{}}\PY{l+s+s1}{guido}\PY{l+s+s1}{\PYZsq{}}\PY{p}{]} \PY{o}{=} \PY{l+m+mi}{4127}
        \PY{o}{\PYZgt{}\PYZgt{}}\PY{o}{\PYZgt{}} \PY{n}{tel}
\end{Verbatim}

    \begin{Verbatim}[commandchars=\\\{\}]
{\color{incolor}In [{\color{incolor} }]:} \PY{o}{\PYZgt{}\PYZgt{}}\PY{o}{\PYZgt{}} \PY{n}{tel}\PY{p}{[}\PY{l+s+s1}{\PYZsq{}}\PY{l+s+s1}{jack}\PY{l+s+s1}{\PYZsq{}}\PY{p}{]}
\end{Verbatim}

    \begin{quote}
\begin{quote}
\begin{quote}
del tel{[}`sape'{]} tel{[}`irv'{]} = 4127 tel
\end{quote}
\end{quote}
\end{quote}

    \begin{Verbatim}[commandchars=\\\{\}]
{\color{incolor}In [{\color{incolor} }]:} \PY{o}{\PYZgt{}\PYZgt{}}\PY{o}{\PYZgt{}} \PY{n+nb}{list}\PY{p}{(}\PY{n}{tel}\PY{o}{.}\PY{n}{keys}\PY{p}{(}\PY{p}{)}\PY{p}{)}
\end{Verbatim}

    \begin{Verbatim}[commandchars=\\\{\}]
{\color{incolor}In [{\color{incolor} }]:} \PY{o}{\PYZgt{}\PYZgt{}}\PY{o}{\PYZgt{}} \PY{n+nb}{sorted}\PY{p}{(}\PY{n}{tel}\PY{o}{.}\PY{n}{keys}\PY{p}{(}\PY{p}{)}\PY{p}{)}
\end{Verbatim}

    \begin{Verbatim}[commandchars=\\\{\}]
{\color{incolor}In [{\color{incolor} }]:} \PY{o}{\PYZgt{}\PYZgt{}}\PY{o}{\PYZgt{}} \PY{l+s+s1}{\PYZsq{}}\PY{l+s+s1}{guido}\PY{l+s+s1}{\PYZsq{}} \PY{o+ow}{in} \PY{n}{tel}
\end{Verbatim}

    \begin{Verbatim}[commandchars=\\\{\}]
{\color{incolor}In [{\color{incolor} }]:} \PY{o}{\PYZgt{}\PYZgt{}}\PY{o}{\PYZgt{}} \PY{l+s+s1}{\PYZsq{}}\PY{l+s+s1}{jack}\PY{l+s+s1}{\PYZsq{}} \PY{o+ow}{not} \PY{o+ow}{in} \PY{n}{tel}
\end{Verbatim}

    Le constructeur dict () construit des dictionnaires directement à partir
de séquences de paires clé-valeur:

    \begin{Verbatim}[commandchars=\\\{\}]
{\color{incolor}In [{\color{incolor} }]:} \PY{o}{\PYZgt{}\PYZgt{}}\PY{o}{\PYZgt{}} \PY{p}{\PYZob{}}\PY{n}{x}\PY{p}{:} \PY{n}{x}\PY{o}{*}\PY{o}{*}\PY{l+m+mi}{2} \PY{k}{for} \PY{n}{x} \PY{o+ow}{in} \PY{p}{(}\PY{l+m+mi}{2}\PY{p}{,} \PY{l+m+mi}{4}\PY{p}{,} \PY{l+m+mi}{6}\PY{p}{)}\PY{p}{\PYZcb{}}
\end{Verbatim}

    Lorsque les clés sont des chaînes simples, il est parfois plus facile de
spécifier des paires en utilisant des arguments de mots clés:

    \begin{Verbatim}[commandchars=\\\{\}]
{\color{incolor}In [{\color{incolor} }]:} \PY{o}{\PYZgt{}\PYZgt{}}\PY{o}{\PYZgt{}} \PY{n+nb}{dict}\PY{p}{(}\PY{n}{sape}\PY{o}{=}\PY{l+m+mi}{4139}\PY{p}{,} \PY{n}{guido}\PY{o}{=}\PY{l+m+mi}{4127}\PY{p}{,} \PY{n}{jack}\PY{o}{=}\PY{l+m+mi}{4098}\PY{p}{)}
\end{Verbatim}

    \section{Référence}\label{ruxe9fuxe9rence}

https://docs.python.org/3.3/tutorial/datastructures.html


    % Add a bibliography block to the postdoc
    
    
    
    \end{document}
